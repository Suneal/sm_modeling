\section{Related Work}
\label{sec:related-work}
In terms of leveraging NLP tools to perform analysis of the security of smart home automation, work by Tian et al. (SmartAuth)~\cite{tzl+17} and Ding et al. (IoTMon)~\cite{dh18} are similar to our paper. However, there are some key differences. First, we target different region of the home automation system. While both Tian and IoTmon are using NLP techniques to analyze IoT application, routines created by third-party developers, descriptions to identify physical channel in IoT platform, our goal is leveraging NLP techniques to analyze the \textit{user-driven} routines. Second, we utilize the result of NLP for a different purpose. While they integrated NLP techniques to devise static analysis system to detect different security violations, we leverage NLP techniques for a quantitative analysis, and call for a motivation to prioritize routines that are used more common by real users. In addition, we investigated and propose a method that enables the analysis of user-driven routines, which are less common research area.
The closest relevant work is by Surbatovich et al.~\cite{sab+17}. They also focused on the idea of the need to examine \textbf{both} user created and developer-created If-This-Then-That (IFTTT) recipes, analogous to what an IoT app is. However, we differ in the problem scope and domain. IFTTT is a platform that connects external services like Instagram or Twitter to smart home devices. For example, user can create an IFTTT recipe that can turn on a smart light when the pizza delivery guy is on his way. This work by Surbatovich et al. focuses on exposing security and privacy ramifications of IFTTT recipes. In contrast, our goal is to enable analysis of user-driven routine in home automation, which limits our scope to events within users' home.

%sequences composed from multiple routines.


%Prior work has proposed a diverse set of security systems and analyses for smart homes. 
%In terms of analysis and enforcement, the natural perspective provided by \tool is complementary to the diverse set of security systems and analyses for smart homes proposed by prior work.
%For instance, consider ContexIoT~\cite{jcw+17}, an access control system that prompts the user for authorization whenever it identifies the use of sensitive operations (\eg unlocking the door) in new contexts.
%To measure the frequency of user-prompts (\ie which affect usability), ContexIoT uses random event sequences generated by fuzz testing of IoT apps.
%The test cases predicted by \tool would provide a more realistic input, 
%%relative to random event sequences, 
%leading to a representative evaluation for such systems.
%Similarly, IoTSAN~\cite{nsq+18}, a system that uses model checking to analyze IoT setups for safety, also uses random events for evaluation, and may similarly benefit from \tool. 
%Additionally, \tools\ {\em down} flavor can directly contribute to adversarial benchmarks such as IoTBench~\cite{cmt18}, for effective evaluation of future work.
%
%Finally, policy-based enforcement for the smart home (\eg ProvThings~\cite{whbg18} and IoTGuard~\cite{ctm19}) can significantly benefit from \tools semi-automated approach for creating policies (Sec.~\ref{sec:security_researchers}).
%For instance, ProvThings~\cite{whbg18} is a provenance tracker for the SmartThings platform, which can also enforce policies. 
%However, it relies on manually-curated policies that are generated from already-occurred security incidents, or domain expertise. 
%As described in Section~\ref{sec:case_studies}, \tool can help security experts create effective policies arising from natural regularities in user-driven home automation sequences, without significant manual effort.
%Analysis that relies on security or safety policies, such as Soteria~\cite{cmt18} and IoTSan~\cite{nsq+18}, can similarly benefit from \tool. 

%\myparagraph{2. Treating software as a natural language} 
