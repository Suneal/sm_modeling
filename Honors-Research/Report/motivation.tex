\section{Motivation}
\label{sec:motivation}

\begin{table*}[t]
\centering
\scriptsize
\caption{{\small Examples of complex routine created by users.}}
\label{tbl:complex-routines}
\begin{tabular}{l|l}
\Xhline{2\arrayrulewidth}
{\bf No.}& {\bf Trigger}\\
\Xhline{2\arrayrulewidth}
1 & (The presence sensor is not detected for more than 1 hour OR the time is between 1AM and 5:30AM) AND the door is unlocked \\
2 & The blinds are closed AND the time is between 10AM and 6PM AND the motion sensor does not detect motion\\
3 & The mode is Vacation AND Gas Sensor odorLevel reach "X" OR Water Leak Detector is wet OR carbonMonoxide Level is above "Y"\\
\Xhline{2\arrayrulewidth}
\end{tabular}
\end{table*}



In order to obtain unconstrained user-driven routines from users, participants provided routine in plain English text. While this approach provides maximum flexibility and reflects \textit{direct} users' requirements, there are several complications that make \textit{automatic} analysis of user-driven routines a non-trivial task.
We can summarize the complications as the follows:
\begin{itemize}
    \item \textbf{Complication 1:} Many desired behaviors for one smart home automation can involve more than one smart device. Specifically, some trigger or action or a combination of both might can incorporate multiple devices.

    \item \textbf{Complication 2:} In practice, users often combine multiple triggers in a large number of unique ways. For example, user can specify to turn off the security camera when user is at home AND it is morning. 

    \item \textbf{Complication 3:} Since routines are written in plain text, key device attributes and state variables are often \textit{implicit}. For example, one user created the following routine: If the temperature is low, then change the thermostat mode to cold. In this case, there is no references of a device name in the trigger part of the routine. 
\end{itemize}

\subsection{Motivating Examples}
We introduce three noticeable samples of user-driven triggers for exposition and illustration (Table.~\ref{tbl:complex-routines}). Each of these routines demonstrated one or more complications describe above. For example, the first user-driven trigger is:
\noindent\fbox{%
    \parbox{\linewidth}{%
(The presence sensor is not detected for more than 1 hour OR the time is between 1AM and 5:30AM) AND the door is unlocked.
	}
}

To begin, this trigger requires multiple devices (Challenge 1) separated by conjunction ``AND'' and ``OR''. In addition, although the fact that the trigger ``door is unlocked'' is referring to door lock is obvious to us, a traditional script will not be able to recognize it since smart door does not 'exist', and the only commercially available parts relating to door are door lock, or doorbell. 

Based on these observations, a general static analysis script without using NLP would not be \textit{effective} in automatically gather and output the desired intermediate representations of user-driven routines. Therefore, we propose the following framework that leverages state-of-art NLP toolkits and techniques to perform the transformation of user-driven routine to intermediate representations. (Figure.~\ref{fig:nlp-framework})

