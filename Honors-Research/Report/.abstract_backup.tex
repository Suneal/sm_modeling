\begin{abstract}

    Routines are trigger-action programs that user create when automating their home. While routine itself offers a wide variety of utilities with a simple conditional format: if \textit{trigger} then \textit{action}, misconfigured or conflicting routines can be potentially dangerous. To fully analyze home automation, we not only require to examine IoT apps, routine created by thrid-party developers, but also examine routines from users' perspective, \textit{user driven routine}.
    In addition, analysis of user driven routines is non-trivial due the complex nature of natural language itself. In this paper, we propose an approach that uses Natural Language Processing (NLP) to transform user driven routines expressed in natual language into intermediate format, which can later be automatically analyzed. 
    In specific, we extracted device names, device capabilities, and device states for both trigger and action from user driven routines.
    We evaluate this approach with a dataset of 250 user driven routines from a concurrent work and produced an accuracy of 80.64\%, \xy\%, \xy\% respectively.
%through an empirical evaluation of 250 routines from 37 users, we propose an approach using Natural Language Processing tools to enable the analysis of user driven routines.
    %We evaluate this approach by trying to \textit{automatically} gather device names, device capabilities, and device variables for each routine specified in the survey. In the end, the approach produced an accuracy of 80.64\% in gathering the device name.
    This demonstrate a reliable and promising approach in extracting key elements from \textit{natural} user driven routines. In addition, we provide some insights to a series of questions regarding to composition and flavor of routines from end-users' perspective. Finally, we describe some challenges to this approach and propose several potential improvements for future work.
\end{abstract}

